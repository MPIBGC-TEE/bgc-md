%% LyX 2.0.5.1 created this file. For more info, see http://www.lyx.org/.
%% Do not edit unless you really know what you are doing.
\documentclass[10pt]{beamer}
\setcounter{secnumdepth}{3}
\setcounter{tocdepth}{3}
\usepackage{url}
\ifx\hypersetup\undefined
  \AtBeginDocument{%
    \hypersetup{unicode=true,pdfusetitle,
 bookmarks=true,bookmarksnumbered=false,bookmarksopen=false,
 breaklinks=false,pdfborder={0 0 0},backref=false,colorlinks=false}
  }
\else
  \hypersetup{unicode=true,pdfusetitle,
 bookmarks=true,bookmarksnumbered=false,bookmarksopen=false,
 breaklinks=false,pdfborder={0 0 0},backref=false,colorlinks=false}
\fi
\usepackage{breakurl}

\makeatletter

%%%%%%%%%%%%%%%%%%%%%%%%%%%%%% LyX specific LaTeX commands.
\providecommand{\LyX}{\texorpdfstring%
  {L\kern-.1667em\lower.25em\hbox{Y}\kern-.125emX\@}
  {LyX}}

%%%%%%%%%%%%%%%%%%%%%%%%%%%%%% Textclass specific LaTeX commands.
 % this default might be overridden by plain title style
 \newcommand\makebeamertitle{\frame{\maketitle}}%
 \AtBeginDocument{
   \let\origtableofcontents=\tableofcontents
   \def\tableofcontents{\@ifnextchar[{\origtableofcontents}{\gobbletableofcontents}}
   \def\gobbletableofcontents#1{\origtableofcontents}
 }
 \def\lyxframeend{} % In case there is a superfluous frame end
 \long\def\lyxframe#1{\@lyxframe#1\@lyxframestop}%
 \def\@lyxframe{\@ifnextchar<{\@@lyxframe}{\@@lyxframe<*>}}%
 \def\@@lyxframe<#1>{\@ifnextchar[{\@@@lyxframe<#1>}{\@@@lyxframe<#1>[]}}
 \def\@@@lyxframe<#1>[{\@ifnextchar<{\@@@@@lyxframe<#1>[}{\@@@@lyxframe<#1>[<*>][}}
 \def\@@@@@lyxframe<#1>[#2]{\@ifnextchar[{\@@@@lyxframe<#1>[#2]}{\@@@@lyxframe<#1>[#2][]}}
 \long\def\@@@@lyxframe<#1>[#2][#3]#4\@lyxframestop#5\lyxframeend{%
   \frame<#1>[#2][#3]{\frametitle{#4}#5}}

%%%%%%%%%%%%%%%%%%%%%%%%%%%%%% User specified LaTeX commands.
\usetheme{PaloAlto}

\makeatother

\usepackage{Sweave}
\begin{document}

\begin{Schunk}
\begin{Sinput}
> opts_chunk$set(fig.path='figure/beamer-',fig.align='center',fig.show='hold',size='footnotesize')
\end{Sinput}
\end{Schunk}


\title[knitr, Beamer, and FragileFrame]{A Minimal Demo of knitr with Beamer and Fragile Frames}


\author{Yihui Xie%
\thanks{I thank Richard E. Goldberg for providing this demo.%
}}

\makebeamertitle

\lyxframeend{}


\lyxframeend{}\lyxframe{Background}
\begin{itemize}
\item The \textbf{knitr}\textbf{\emph{ }}package allows you to embed R code
and figures in \LaTeX{} documents

\begin{itemize}
\item It has functionality similar to Sweave but looks nicer and gives you
more control
\end{itemize}
\item If you already have Sweave working in \LyX{}, getting \textbf{knitr}
to work is trivial

\begin{enumerate}
\item Install the \textbf{knitr} package in \emph{R}
\item Read \url{http://yihui.name/knitr/demo/lyx/}
\end{enumerate}
\item If you use Sweave or \textbf{knitr} with Beamer in \LyX{}, you probably
use the \emph{Beamer Fragile} module%
\footnote{\url{http://www.lyx.org/trac/ticket/7273}%
} too. Let's see if \textbf{knitr} works with Beamer in this small
demo.
\end{itemize}

\lyxframeend{}\section{First Test}

\begin{frame}
[fragile]
\frametitle{First Test}

OK, let's get started with just some text:

